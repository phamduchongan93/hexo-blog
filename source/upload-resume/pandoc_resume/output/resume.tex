% Copyright 2013 Christophe-Marie Duquesne <chmd@chmd.fr>
% Copyright 2014 Mark Szepieniec <http://github.com/mszep>
% 
% ConText style for making a resume with pandoc. Inspired by moderncv.
% 
% This CSS document is delivered to you under the CC BY-SA 3.0 License.
% https://creativecommons.org/licenses/by-sa/3.0/deed.en_US

\startmode[*mkii]
  \enableregime[utf-8]  
  \setupcolors[state=start]
\stopmode

\setupcolor[hex]
\definecolor[titlegrey][h=757575]
\definecolor[sectioncolor][h=397249]
\definecolor[rulecolor][h=9cb770]

% Enable hyperlinks
\setupinteraction[state=start, color=sectioncolor]

\setuppapersize [A4][A4]
\setuplayout    [width=middle, height=middle,
                 backspace=20mm, cutspace=0mm,
                 topspace=10mm, bottomspace=20mm,
                 header=0mm, footer=0mm]

%\setuppagenumbering[location={footer,center}]

\setupbodyfont[11pt, helvetica]

\setupwhitespace[medium]

\setupblackrules[width=31mm, color=rulecolor]

\setuphead[chapter]      [style=\tfd]
\setuphead[section]      [style=\tfd\bf, color=titlegrey, align=middle]
\setuphead[subsection]   [style=\tfb\bf, color=sectioncolor, align=right,
                          before={\leavevmode\blackrule\hspace}]
\setuphead[subsubsection][style=\bf]

\setuphead[chapter, section, subsection, subsubsection][number=no]

%\setupdescriptions[width=10mm]

\definedescription
  [description]
  [headstyle=bold, style=normal,
   location=hanging, width=18mm, distance=14mm, margin=0cm]

\setupitemize[autointro, packed]    % prevent orphan list intro
\setupitemize[indentnext=no]

\setupfloat[figure][default={here,nonumber}]
\setupfloat[table][default={here,nonumber}]

\setuptables[textwidth=max, HL=none]
\setupxtable[frame=off,option={stretch,width}]

\setupthinrules[width=15em] % width of horizontal rules

\setupdelimitedtext
  [blockquote]
  [before={\setupalign[middle]},
   indentnext=no,
  ]


\starttext

\section[title={An Pham},reference={an-pham}]

\startplacetable[location=none]
\startxtable
\startxtablebody[body]
\stopxtablebody
\startxtablefoot[foot]
\startxrow
\startxcell Deploying automation for Software Devlopment with Devops and
Gitops approach. \stopxcell
\stopxrow
\stopxtablefoot
\stopxtable
\stopplacetable

\subsection[title={Education},reference={education}]

\startdescription{2018-2019 (expected)}
  {\bf Bachelor, Mangement Information System }; University of Boston

  {\em Major : System Administrator - Information Technoloogy}
\stopdescription

\subsection[title={Experience},reference={experience}]

{\bf System Engineering Virtual Intern}

Open Learning Exchange, MA, USA Duration: Sept 2019 - Present, 2020 Team
Size: 12-15 Role Played: QA engineer

\startitemize[packed]
\item
  Build and test jitsi meeting application
\item
  Provide whitebox and blackbox testing for raspbian images.
\item
  Test the CI/CD for container
\item
  Code review for bash programs
\item
  Debug and program wireless system
\stopitemize

{\bf AV Techology Support}

Umass Boston, Boston, MA - Perform data entry on daily basis. - Update
laptop and Workstation of class rooms. - Troubleshoot tickets from ITSM.
- Responsible for reporting incidents within buildings. - Perform
installation for Zoom Conference Software.

Software has been used * Ticket system - Service Now * Office 365

\subsection[title={Project},reference={project}]

1 - System Engineer Virtual Intern Skills Used: Rasbian, Docker,
Vagrant, Bash, NPM.

\starttyping
Create debian based images that can be controled by android remote apps. The system is used to support a web based e-learning system. The scope of the projects is high and maintained by a experience devops team.  
\stoptyping

\startitemize[packed]
\item
  QA Engineer - add
\item
  Docker Pipeline - Link
\stopitemize

2 - SD Card Burner

3 - Blogging about KVM technology. - Build virtual machine for system
testing. - Containerize web apps to work with docker.

\subsection[title={Technical
Experience},reference={technical-experience}]

\startdescription{VM Builder}
  For items which don't have a clear time ordering, a definition list
  can be used to have named items.

  \startitemize[packed]
  \item
    These items can also contain lists, but you need to mind the
    indentation levels in the markdown source.
  \item
    Second item.
  \stopitemize
\stopdescription

\startdescription{Open Source}
  SD Card Burner
  \useURL[url1][sd-burner.netify.com][][https://github.com/ole-vi/]\from[url1]
\stopdescription

\startdescription{Programming Languages}
  {\bf Bash} Here, we have an itemization, where we only want

  {\bf Python:} Description of your experience with second-lang,

  {\bf LAMP stack} We both know this one's pushing it.

  Nodejs cli module
\stopdescription

\subsection[title={Skills},reference={skills}]

\startitemize[packed]
\item
  Documentation style
  \startitemize[packed]
  \item
    English
  \item
    Hexo, markdown static webs
  \item
    CI/CD DocOps
  \stopitemize
\item
  Automation and Pipeline
  \startitemize[packed]
  \item
    Automated Build: Jenkins, Gits
  \item
    Automated Testing: BATs
  \item
    Docker webhook
  \stopitemize
\stopitemize

\startblockquote
\useURL[url2][mailto:an.pham001@umb.edu][][an.pham001@umb.edu]\from[url2]
• +857 308 6648 •\crlf
Dorchester, MA 02125
\stopblockquote

\stoptext
